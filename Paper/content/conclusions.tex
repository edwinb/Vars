%\section{Discussion}

\section{Related Work}

This paper builds on previous work on algebraic effects in
Idris~\citep{brady-eff2013,brady-tfp14}, and the implementation of
\texttt{STrans} follows many of the ideas used in these earlier
implementations. However, this earlier work had several limitations, most
importantly that it was not possible to implement one effectful API in
terms of others, and that it was difficult to describe the relationship
between separate resources, such as the fact that \texttt{accept} creates
a new resource in a specific initial state. The work in the present paper is
influenced in particular by previous work on Separation Logic, Linear Types 
and Session Types, and has connections with other type systems and tools
for formal verification. In this section we discuss these and other
connections.

\emph{\textsf{Separation Logic and Indexed Monads:}}
%
The representation of state transition systems using dependent types owes much
to Atkey's study of indexed monads~\citep{atkey-param}, and McBride's
implementation of dynamic interaction in Haskell~\citep{mcbride-kleisli}.  The
state transitions given by operations in \states{}, like in this previous
work on indexed monads, are reminiscent of Hoare Triples~\citep{hoarelogic}. 
Separation Logic~\citep{reynolds2002}, an extension of Hoare Logic, 
permits local reasoning by allowing a heap to be split into two disjoint
parts, like the \texttt{call} function in
\states{}. This has previously been implemented as Ynot, an axiomatic extension
to Coq~\citep{ynot08}. 
Separation Logic allows us to reason about the state of a program's heap.  The
key distinction between this previous work and \states{} is that we reason
about individual resources, such as files and network sockets, rather than
about the entire heap, and list these resources in a context.
While Separation Logic can prove more complex properties, our
approach has a clean embedding in the host language, which
leads to readable types and, because resources are combined in predictable
ways, minimises the proof obligations for the programmer. Indeed, the examples
in Section~\ref{sect:examples} require \emph{no} additional effort
from the programmer to prove that resources are used correctly.

% \citep{Chlipala2015}

\emph{\textsf{Linear Types:}}
%
Indexed monads, as used in \states{}, are a way of encoding
linearity~\citep{wadler-linear,Abramsky1993} 
in a dependently typed host language. Recent work by~\citet{neelk2015}
and by~\citet{McBride2016} has shown ways of integrating linear and
dependent types, and the latter has been implemented as an experimental
extension to Idris.
Our experience has been that linear types can provide the same guarantees
as the \states{} library, but that, subjectively, they have a more ``low-level''
feel than the indexed monad approach. A particular notational inconvenience is
the need for a function which updates a linear resource to return a new 
reference to the resource, a detail which is abstracted away by the indexed
monad. Nevertheless, we hope to explore the connection further in future work.
In particular, linear dependent types may give us an efficienct
method for implementing \states{}.

\emph{\textsf{Types and State Machines:}}
%
Earlier work has recognised the importance of state transition systems
in describing applications~\citep{statecharts}.
%
In this paper, we have used \states{} to describe systems in terms
of state transitions on resources,
both at the level of external resources like network sockets and at
the application level.
The problem of reasoning about protocols has previously been
tackled using special purpose type systems~\citep{Walker2000}, by creating
DSLs for resource management~\citep{Brady2010a}, or with
Typestate~\citep{Aldrich2009,Strom1986}. In these approaches, however,
it is difficult to compose systems from multiple resources or to implement a
resource in terms of other resources.  In \states{}, we can combine resources
by using additional interfaces, and implement those interfaces in terms of
other lower level resources.

\emph{\textsf{Types for Communication:}}
%
A strong motivation for the work in the present paper is to be able to
incorporate a form of Session Types~\citep{Honda93,Honda08} into dependently
typed applications, while also supporting other stateful components.  
Session Types describe the state of a communication channel, 
and recent work has shown the correspondence between propositions and 
sessions~\citep{propositions-sessions} and how to implement this in a
programming language~\citep{Lindley2015}. 
More recently,~\citet{Toninho2016} have shown how to increase the
expressivity of multi-party session to capture invariants on data.
We expect
to be able to encode similar properties in \states{} and in doing so, be
able to encode security properties of communicating systems with
dependent types, following~\citep{Guenot2015}.
Type systems in modern programming languages, such as Rust~\citep{session-rust}
and Haskell~\citep{session-haskell}, are strong enough to support Session
Types, and by describing communication protocols in types, we expect to be able
to use the type system to verify correctness of implementations of security
protocols~\citep{gordon2003authenticity,sewell-tls}.


%Also Dijkstra Monads~\cite{Ahman2017}

\emph{\textsf{Effects and Modules:}}
%
Algebraic effects~\citep{Plotkin2009,Bauer} are increasingly being proposed as
an alternative to monad transformers for structuring Haskell
applications~\citep{KiselyovEffects,handlers2013}.
Earlier work in Idris~\cite{brady-eff2013,brady-tfp14} has used a system
based on algebraic effects to track resource state, like \states{} in
this paper. 
Koka~\cite{Leijen2017} uses row polymorphism to describe the allowed effects
of a function, which is similar to the \texttt{Context} of an \states{}
program in that a function's type lists only the parts of the overall state it
needs.

In \states{}, rather than using effects and handlers, we use
\emph{interfaces} to constrain the operations on a resource, and provide
\emph{implementations} of those interfaces. This does not have the full
power of handlers of algebraic effects---in particular, we cannot reorder
handlers to implement exception handling or other control flow---but it
does allow us to express the state transitions precisely. Indeed, we do not
want handlers to be \emph{too} flexible: for example, a handler which
implements non-determinism by executing an operation several times would
violate linear access to a resource.
%
Interfaces in Idris are similar to Haskell type classes, but with support
for \emph{named} overlapping instances. This means that we can give multiple
different interpretations to an interface in different implementations, and
in this sense they are similar in power to ML modules~\cite{Dreyer2005}.
Interfaces in Idris are first class, meaning that implementations can be
calculated by a function, and thus provide the same power as first-class
modules in 1ML~\citet{rossberg2015}.

\section{Conclusions and Further Work}

We have shown how to describe APIs in terms of type-dependent state transition
systems, using a library \states{} and evaluated the library by presenting
several diverse examples which show how \states{} can be used in practice.
In particular, we have shown how to design larger scale systems by composing
state machines both horizontally (that is, using multiple state machines in the
same function definition) and vertically (that is, implementing the behaviour
of a state machine using other lower level state machines). 

A significant strength of this approach to structuring 
programs is that state machines are ubiquitous, if implicit, in realistic APIs,
especially when dealing with external resources like surfaces on which to draw
graphics, endpoints in network communication, and databases.  The \states{}
library makes these state machines explicit in types, meaning
that we can be sure by type checking that a program correctly follows the state
machine.  As the \texttt{Sockets} interface shows, we can give precise types to
an existing API (in this case, the POSIX sockets API), and use the operations
in more or less the same way, with additional confidence in their correctness.
%
Furthermore, as we briefly discussed in Section~\ref{sect:getdata}, and 
saw throughout the paper, we can use \emph{interactive}, type-driven, program
development and see the internal state of a program at any point during
development.

Since \states{} is parameterised by an underlying computation context, which is
most commonly a monad, it is a monad transformer.  Also, an instance of
\states{} which preserves states is itself a monad, so \states{} programs can
be combined with monad transformers, and therefore are compatible with a common
existing method of organising large scale functional programs.

There are several avenues for further work.  For example, we have not discussed
the efficiency of our approach in this paper. There is a small overhead due to
the interpreter for \states{} but we expect to be able to eliminate this using
a combination of partial evaluation~\citet{scrap-engine} and a finally tagless
approach to interpretation~\citet{Carette2009}. We may also be able to use
linear types for the underlying implementation~\citet{McBride2016} using an
experimental extension to Idris.

Most importantly, however, we believe there are several applications to
this approach in defining security protocols, and verified implementation
of distributed systems. For the former, security protocols follow a clearly
defined series of steps and any violation can be disastrous, causing
sensitive data to leak.
%
For the latter, we are currently developing an implementation of Session
Types~\citet{Honda93,Honda08} embedded in Idris, generalising the random
number server presented in Section~\ref{sect:randserver}.

The \states{} library provides a generic interface for implementing a useful
pattern in dependently typed program in such a way that it is \emph{reusable}
by application developers. State is everywhere, and introduces complexity
throughout applications. Dependently typed purely functional programming gives
us the tools we need to keep this complexity under control.
