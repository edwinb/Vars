\section{Implementation}

\label{sect:implementst}

As we have seen, \texttt{STrans} allows us to write programs which
describe sequences of state transitions, creating and destroying resources
as necessary. In this section, we will describe some implementation details
of \texttt{STrans}. In particular, we will describe how a \texttt{Context}
corresponds to an environment containing concrete values at run time, and
see the concrete definition of \texttt{STrans} itself. We will show
the mechanism for calling subprograms with smaller sets of resources, and
show how to build \emph{composite} resources, consisting of a list of
independent resources.  Finally, we will briefly describe how Idris' error
reflection mechanism~\citep{christiansen-thesis} lets us rewrite type error
messages to describe errors in terms of the problem domain, rather than in
terms of the implementation language.

\subsection{Environments and Execution}

The indices of \texttt{STrans} describe how a sequence of operations affects
the types of elements in a context. Correspondingly, when we \texttt{run}
an \texttt{STrans} program, we need to keep track of the \emph{values}
of those elements in an environment, defined as follows:

\small
\begin{code}
data Env : Context -> Type where
     Nil : Env []
     (::) : ty -> Env xs -> Env ((lbl ::: ty) :: xs)
\end{code}
\normalsize

Then, when running a program, we provide an \emph{input} environment,
and a continuation which explains how to process the result of the program
and the output environment, whose type is calculated from the result:

\small
\begin{code}
runST : Env invars -> STrans m a invars outfn -> 
        ((x : a) -> Env (outfn x) -> m b) -> m b
\end{code}
\normalsize

At the top level, we can \texttt{run} a program with no resources on
input and output:

\small
\begin{code}
run : Applicative m => ST m a [] -> m a
run prog = runST [] prog (\res, _ => pure res)
\end{code}
\normalsize

The \texttt{Applicative} constraint on \texttt{run} is required so that
we can inject the result of the computation into the underlying computation
context \texttt{m}. In the special case that \texttt{m} is the identity
function, we can return the result directly instead:

\small
\begin{code}
runPure : ST id a [] -> a
runPure prog = runST [] prog (\res, _ => res)
\end{code}
\normalsize

% Specifically IO because we only want to allow this in concrete settings.
% Idea of Env is that only 'run' can access it but this escape is handy
% for 'fork'.

Finally, in some cases we might want to execute an \texttt{STrans} program
with an initial environment and read the resulting environment, which we can
achieve using \texttt{runWith} provided that we are executing the program in
the \texttt{IO} monad.

\small
\begin{code}
runWith : Env ctxt -> STrans IO a ctxt (\res => ctxtf res) -> 
          IO (res ** Env (ctxtf res))
runWith env prog = runST env prog (\res, env' => pure (res ** env'))
\end{code}
\normalsize

It is important to restrict this to a specific monad, rather than allowing
\texttt{runWith} to be parameterised over any monad \texttt{m} like
\texttt{run}.
The reason is that the intention of \texttt{STrans} is to control all accesses
to state via \texttt{read} and \texttt{write}, but \texttt{runWith} gives
us a convenient ``escape hatch'' if we need more flexibility to modify the
environment in a concrete
\texttt{IO} setting. We will need this when implementing asynchronous 
programs in Section~\ref{sect:async}. By restricting \texttt{runWith} to work
in a \emph{concrete} monad, we know that it is at least impossible to use it
in programs which are written in a generic context. For example, we saw
\texttt{writeToFile} earlier:

\small
\begin{code}
writeToFile : (FileIO m, DataStore m) => (h : Var) -> (st : Var) ->
              ST m () [h ::: File {m} Write, st ::: Store {m} LoggedIn]
\end{code}
\normalsize

We know that it is impossible for \texttt{writeToFile} to call
\texttt{runWith}, and possibly introduce new items into the environment,
because it is parameterised over some monad \texttt{m}.

\subsection{Defining \texttt{STrans}}

\texttt{STrans} itself is defined as an algebraic data type, describing
operations for reading, writing, creating and destroying resources,
and sequencing stateful operations. Additionally, there are operations for
manipulating the context in some more advanced ways. The complete definition
is shown in Listing~\ref{fig:stransdef}.

\small
\begin{code}[float=h, frame=single,caption={The complete definition of
\texttt{STrans} as an Idris data type},label=fig:stransdef]
data STrans : (m : Type -> Type) -> (ty : Type) -> 
              Context -> (ty -> Context) -> Type where
     Pure : (result : ty) -> STrans m ty (out_fn result) out_fn
     Bind : STrans m a st1 st2_fn ->
            ((result : a) -> STrans m b (st2_fn result) st3_fn) ->
            STrans m b st1 st3_fn
     Lift : Monad m => m t -> STrans m t ctxt (const ctxt)
     New : (val : state) -> 
           STrans m Var ctxt (\lbl => (lbl ::: State state) :: ctxt)
     Delete : (lbl : Var) -> (prf : InState lbl (State st) ctxt) ->
              STrans m () ctxt (const (drop ctxt prf))
     DropSubCtxt : (prf : SubCtxt ys xs) -> STrans m (Env ys) xs (const (kept prf))
     Split : (lbl : Var) -> (prf : InState lbl (Composite vars) ctxt) ->
             STrans m (VarList vars) ctxt 
                   (\vs => mkCtxt vs ++ updateCtxt ctxt prf (State ()))
     Combine : (comp : Var) -> (vs : List Var) ->
               (prf : VarsIn (comp :: vs) ctxt) ->
               STrans m () ctxt (const (combineVarsIn ctxt prf))
     Call : STrans m t ys ys' -> (ctxt_prf : SubCtxt ys xs) ->
            STrans m t xs (\res => updateWith (ys' res) xs ctxt_prf)
     Read : (lbl : Var) -> (prf : InState lbl (State ty) ctxt) ->
            STrans m ty ctxt (const ctxt)
     Write : (lbl : Var) -> (prf : InState lbl ty ctxt) -> (val : ty') ->
             STrans m () ctxt (const (updateCtxt ctxt prf (State ty')))
\end{code}
\normalsize

The operations we have described so far, in Section~\ref{sect:statemachines},
are implemented by using the corresponding constructor of \texttt{STrans}.
The main difference is that proof terms (such as the \texttt{prf} arguments
of \texttt{Delete} and \texttt{Read}) are explicit, rather than marked
as implicit with \texttt{auto}.
%
There are four constructors we have not yet encountered. These are
\texttt{Lift}, \texttt{DropSubCtxt}, \texttt{Split} and \texttt{Combine}:

\begin{itemize}
\item \texttt{Lift} allows us to use operations in the underlying monad
\texttt{m}. We will need this to implement interfaces in terms of existing
monads.
\item \texttt{DropSubCtxt} allows us to remove a subset of resources, and
returns their values as an environment.  The output context is calculated using
a function \texttt{kept}, which returns the resources which are not part of the
subset. We will use this to share resources between multiple threads in
Section~\ref{sect:async}.
\item \texttt{Split} and \texttt{Combine} allow us to work with
\emph{composite} resources. We will discuss these shortly in
Section~\ref{sect:splitcomb}, and we will need them to be able to implement one
stateful interface in terms of a collection of others. For example, in
Section~\ref{sect:turtle}, we will implement a small high level graphics API
in terms of resources describing a low level graphics context, and some
internal state.
\end{itemize}

The implementation of \texttt{runST} is by pattern matching on \texttt{STrans},
and largely uses well known implementation techniques for well-typed interpreters
with dependent types~\citep{Pasalic2002,augustsson1999exercise}. Most of
the implementation is about managing the environment, particularly
when interpreting \texttt{Call}, \texttt{Split} and \texttt{Combine}, which
involve restructuring the environment according to the proofs.
%
When we interpret \texttt{New}, we need to provide a new \texttt{Var}.
The \texttt{Var} type is defined as follows, and perhaps surprisingly has only
one value:

\small
\begin{code}
data Var = MkVar
\end{code}
\normalsize

We do not need any more than this because internally all references
to variables are managed using proofs of context membership, such as
\texttt{InState} and \texttt{SubCtxt}. The variable of type \texttt{Var} 
gives a human readable name, and a way of disambiguating resources by Idris
variable names, but by the time we execute a program with \texttt{runST}, these
variables have been resolved into proofs of context membership so we do
not need to construct any unambiguous concrete values.

\subsection{Calling subprograms}

\label{sect:callimpl}

We often need to call subprograms with a \emph{smaller} set of resources
than the caller, as we saw in Section~\ref{sect:multiplests}. 
To do this, we provide a proof that the input resources required by the
subprogram (\texttt{sub}) are a subset of the current input resources
(\texttt{old}):

\small
\begin{code}
Call : STrans m t sub new_f -> (ctxt_prf : SubCtxt sub old) ->
       STrans m t old (\res => updateWith (new_f res) old ctxt_prf)
\end{code}
\normalsize

A value of type \texttt{SubCtxt sub old} is a proof that every element
in \texttt{sub} appears exactly once in \texttt{old}. In other words, it is
a proof that the resources in \texttt{sub} are a subset of those in \texttt{old}.
To show this, we need to be able to show that a specific \texttt{Resource} is
an element of a \texttt{Context}:

\small
\begin{code}
data ElemCtxt : Resource -> Context -> Type where
     HereCtxt : ElemCtxt a (a :: as)
     ThereCtxt : ElemCtxt a as -> ElemCtxt a (b :: as)
\end{code}
\normalsize

Then, a proof of \texttt{SubCtxt sub old} 
essentially states, for each entry in \texttt{old},
either where it appears in \texttt{sub} (using \texttt{InCtxt} and a 
proof of its location using \texttt{ElemCtxt}) or that it is not present in
\texttt{sub} (using \texttt{Skip}):

\small
\begin{code}
data SubCtxt : Context -> Context -> Type where
     SubNil : SubCtxt [] []
     InCtxt : (el : ElemCtxt x ys) -> SubCtxt xs (dropEl ys el) ->
              SubCtxt (x :: xs) ys
     Skip : SubCtxt xs ys -> SubCtxt xs (y :: ys)
\end{code}
\normalsize

A context is a subcontext of itself, so we can say that \texttt{[]} is
a subcontext of \texttt{[]}.
To ensure that there is no repetition in the sub-context, the recursive
argument to \texttt{InCtxt} explicitly states that the resource cannot
appear in the remaining sub-context, using \texttt{dropEl}:

\small
\begin{code}
dropEl : (ys: _) -> ElemCtxt x ys -> Context
dropEl (x :: as) HereCtxt = as
dropEl (x :: as) (ThereCtxt p) = x :: dropEl as p
\end{code}
\normalsize

The type of \texttt{Call} relies on the following function, which
calculates a new context for the calling function, based on the updates
made by the subprogram:

\small
\begin{code}
updateWith : (new : Context) -> (old : Context) -> SubCtxt sub old -> Context
\end{code}
\normalsize

The result of \texttt{updateWith} is the contents of \texttt{new}, adding
those values in \texttt{old} which were previously \texttt{Skip}ped as
described by the proof of \texttt{SubCtxt sub old}.
Correspondingly, there is a function used by \texttt{runST} which rebuilds
an environment after evaluating the \texttt{Call}ed subprogram:

\small
\begin{code}
rebuildEnv : Env new -> Env old -> (prf : SubCtxt sub old) ->
             Env (updateWith new old prf)
\end{code}
\normalsize

When we use \texttt{Call}, the contexts described by 
\texttt{SubCtxt} are known at compile time, from the type of the subprogram
and the state of the \texttt{Context} at the call site. Idris'
built in proof search, which searches constructors up to a 
depth limit until it finds an application which type checks, is strong enough
to find these proofs without programmer intervention, so in the top level
\texttt{call} function we mark the proof argument as \texttt{auto}:

\small
\begin{code}
call : STrans m t sub new_f -> {auto ctxt_prf : SubCtxt sub old} ->
       STrans m t old (\res => updateWith (new_f res) old ctxt_prf)
\end{code}
\normalsize

\subsection{Composite Resources}

\label{sect:splitcomb}

A \emph{composite} resource is built from a list of other resources, and
is essentially defined as a heterogeneous list:

\small
\begin{code}
data Composite : List Type -> Type
\end{code}
\normalsize

If we have a composite resource, we can split it into its
constituent resources, and create new variables for each of those resources,
using the following top level function which is defined using \texttt{Split},
again using proof search to find the label in the context:

\small
\begin{code}
split : (lbl : Var) -> {auto prf : InState lbl (Composite vars) ctxt} ->
        STrans m (VarList vars) ctxt 
                 (\vs => mkCtxt vs ++ updateCtxt ctxt prf (State ()))
\end{code}
\normalsize

A \texttt{VarList} is a list of variable names, one for each resource
in the composite resource. We use \texttt{mkCtxt} to convert it into
a fragment of a context where the composite resource has been split into
independent resources:
  
\small
\begin{code}
VarList : List Type -> Type
mkCtxt : VarList tys -> Context
\end{code}
\normalsize

After splitting the composite resource, the original resource is replaced with
unit. We can see the effect of this in the following code fragment,
in a function which is intended to swap two integers in a composite resource:

\small
\begin{code}
swap : (comp : Var) -> ST m () [comp ::: Composite [State Int, State Int]]
swap comp = do [val1, val2] <- split comp
               ?rest
\end{code}
\normalsize

If we check the type of the hole \texttt{?rest}, we see that \texttt{val1}
and \texttt{val2} are now individual resources:

\small
\begin{code}
rest : STrans m () [val1 ::: State Int, val2 ::: State Int, comp ::: State ()]
            (const [comp ::: Composite [State Int, State Int]])
\end{code}
\normalsize

The type of \texttt{?rest} shows that, on exit, we need a composite resource
again. We can build a composite resource from individual resources using
\texttt{combine}, implemented in terms of the corresponding \texttt{STrans}
constructor:

\small
\begin{code}
combine : (comp : Var) -> (vs : List Var) ->
          {auto prf : InState comp} -> {auto var_prf : VarsIn (comp :: vs) ctxt} ->
          STrans m () ctxt (const (combineVarsIn ctxt var_prf))
\end{code}
\normalsize

Similar to \texttt{SubCtxt}, \texttt{VarsIn} is a proof that every variable
in a list appears once in a context. Then, \texttt{combineVarsIn} replaces
those variables with a single \texttt{Composite} resource. Correspondingly,
in the implementation of \texttt{runST}, \texttt{rebuildVarsIn} updates the
environment:

\small
\begin{code}
rebuildVarsIn : Env ctxt -> (prf : VarsIn (comp :: vs) ctxt) -> 
                Env (combineVarsIn ctxt prf)
\end{code}
\normalsize

Using \texttt{combine}, we can reconstruct the context in \texttt{swap} with
the resources swapped:

\small
\begin{code}
swap : (comp : Var) -> ST m () [comp ::: Composite [State Int, State Int]]
swap comp = do [val1, val2] <- split comp
               combine comp [val2, val1]
\end{code}
\normalsize

\subsection{Improving Error Messages with Error Reflection}

\label{sect:errorreflection}

Helpful error messages make an important contribution to the usability of any
system. We have used Idris' error message reflection~\citep{christiansen-thesis}
to rewrite the errors produced by Idris to explain the relevant part of
specific errors, namely, the preconditions and postconditions on operations.
Consider the following incorrect code fragment, using the data store defined
in Section~\ref{sect:strans} but attempting to read without first logging in:

\small
\begin{code}
badGet : DataStore m => ST m () []
badGet = do st <- connect
            secret <- readSecret st
            ?more
\end{code}
\normalsize

By default, Idris reports the following as part of the error message:

\small
\begin{code}
When checking an application of function Control.ST.>>=:
        Type mismatch between
                STrans m String [st ::: Store LoggedIn]
                       (\result =>
                          [st ::: Store LoggedIn]) (Type of readSecret st)
        and     STrans m String [st ::: Store LoggedOut]
                       (\result => [st ::: Store LoggedIn]) (Expected type)
\end{code}
\normalsize

This includes the relevant information, that the store needs to be
\texttt{LoggedIn} but in fact is \texttt{LoggedOut}, but the important parts
are hidden inside a larger term. However, recognising that errors arising
from running an operation when the preconditions do not match always have
the same form, we can extract the pre- and postconditions from the message,
and rewrite it to the following using an \emph{error handler}:

\small
\begin{code}
Error in state transition:
    Operation has preconditions: [st ::: Store LoggedIn]
    States here are: [st ::: Store LoggedOut]
    Operation has postconditions: \result => [st ::: Store LoggedIn]
    Required result states here are: \result => [st ::: Store LoggedIn]
\end{code}
\normalsize

The full details are beyond the scope of this paper, but they rely on
the following function which inspects a reflected error message, and returns
a new message if the original message has a particular form:

\small
\begin{code}
st_precondition : Err -> Maybe (List ErrorReportPart)
\end{code}
\normalsize

